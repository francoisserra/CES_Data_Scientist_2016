\documentclass{book}

\usepackage[utf8]{inputenc}
\usepackage[T1]{fontenc}
\usepackage[francais]{babel}
\usepackage{graphicx} 
\usepackage{fancyref}
\usepackage{hyperref}
\usepackage{url}


\title{%
  Projet de Sciences des Données \\
  \large Explotation d'images satellites haute-résolution \\pour la prevision d'indicateurs socio-économiques \\
    }

\author{\textsc{Youcef} - \textsc{Kacer}}
\date{25 Aout 2016}

\begin{document}
 
\maketitle

\tableofcontents

\frontmatter
\chapter{Introduction}
Ce document présente le projet de sciences des données que je compte développer, dans le cadre de la validation du CES Data Scientist à Telecom-Paristech \cite{cesds}.\\
Ce projet consiste à exploiter des images satellitaires haute-résolution afin d'en extraire des indicateurs socio-économiques.\\
En effet, évaluer la densité démographique d'un pays, par exemple, peut représenter un co\^{u}t non négligeable en terme de 
recensement. Or utiliser des images aériennes et leurs caractéristiques permettrait de prédire la population présente à moindre co\^{u}t.
En effet, les \og edges \fg des routes et des batiments caractérisent les zones urbaines et donc les zones à forte population, alors que
les champs et les for\^{e}ts caractérisent des zones faiblement peuplées.\\

\mainmatter
\chapter{Images exploitées}

Les images à exploiter proviennent du satellite Landsat 8 de la NASA et sont libres d'accès \cite{landsat8}. Ce satellite scanne tout le globe terrestre 
tous les 16 jours. Ces images permettent donc non seulement d'étudier une zone à un moment donnée mais aussi d'étudier son évolution sur
une période donnée.\\
Ces images sont très riches dans la mesure où elles présentent en tout 11 canaux, 9 dans le visible et 2 dans l'infra-rouge, pour des résolutions 
allant de 15 à 60 mètres. Donc en plus des caractéristiques de formes, le niveau des images doit pouvoir nous renseigner sur 
la nature des materiaux et des objets présents au sein d'une zone (métal ou végétation par exemple), les canaux infra-rouges pourront très certainement
quantifier la présence humaine.\\

\chapter{Méthode d'apprentissage}

L'idée serait de s'intéresser à une certaine zone (un pays par exemple), dont on aurait l'indicateur de densité de population (valeur 
à prédire) pour un grand ensemble de communes du pays.\\
On pourra alors récupérer plusieurs images satellitaires quadrillant ce pays, et attribuer à chacune d'elles
sa valeur de densité de population (on doit pouvoir utiliser la latitude et la longitude d'une image pour retrouver la commune concernée).\\
Ainsi, on récupère un ensemble classique d'images labelisées par sa densité de population.\\
Ensuite, on pourra extraire des descripteurs de ces images (histogramme orienté du gradient \cite{Dalal05histogramsof}, entre autre)
auquels on appliquera un algorithme de regression supervisé (la valeur à prédire, la densité de
population, est plut\^{o}t continue que discrète).\\
On aurait donc un modèle de classification capable de prédire la densité de population d'une zone en fonction d'images satellites.\\
On pourra alors tester la généralisation du classifieur, en s'intéressant à d'autres pays.\\

\chapter{Outils pour le stockage, le calcul et la visualisation}

Les images seront stockées sur HDFS \cite{White:2009:HDG:1717298} (soit en mode \og single-node cluster \fg, soit en mode \og multi-node cluster \fg via le cluster de Telecom ParisTech), cela permettra
d'extraire les descripteurs par Map/Reduce. Ces descripteurs seront aussi stockés de manière distribuée, via une table HBASE afin de 
pouvoir effectuer des requ\^{e}tes et vérifier les valeurs calculées.\\
On utilisera alors la librairie de Machine Learning MLlib de Spark \cite{Meng:2016:MML:2946645.2946679}, dédiée à l'apprentissage sur données distribuées dans HDFS.
Pour la regression, cette librairie ne permet cependant que les regressions linéaires (soit par moindres carrés linéaires, par Ridge ou par Lasso) et les regressions logistiques (pas de SVM).\\
On testera toutes ces méthodes pour en analyser les erreurs en cross-validation.\\
Enfin, pour la visualisation des resultats, on proposera une page html utilisant la librairie D3 i\cite{Jain:2014:DVD:2667432.2667451} pour l'affichage de la carte de densité de population. Cette interface web sera aussi
intéractive que possible afin de permettre des zooms mais aussi l'observation de la densité de population sur une année antérieure.

\clearpage

\backmatter

%\listoftables

%\listoffigures

\bibliographystyle{alpha}
\bibliography{biblio}

\end{document}